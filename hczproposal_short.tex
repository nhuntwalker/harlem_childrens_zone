\documentclass[12pt]{article}
\usepackage{amsmath}
\usepackage{amssymb}
\usepackage[hmargin=0.75in,vmargin=0.5in]{geometry}

\begin{document}
\title{Harlem Children's Zone: Astronomy Course Proposal}
\author{Nicholas Hunt-Walker}
\maketitle

I propose to construct a tour of the contents of the night sky in 6-weeks. The primary goals will be bringing space to ground-level, and making the practice of space science achievable for all. The format will be a walk through the various objects that occupy the vast size ranges of our Universe. We start at the Moon, talking about it's dynamic formation history. We then move through the planets, our Sun, our Galaxy, and out to the massive web of galaxy clusters that comprise our Universe. At each stop on our walk, we will study the details of those objects and get a deeper feel for them. While we discuss each, I will introduce the subtle physics that govern the Universe, and how those physical concepts bring us the knowledge of the night sky that we have today. Most lessons will be accompanied by a demonstration either in person or on the computer. These demonstrations can be lab assignments and software that have either been used by the University of Washington or that I have created myself, depending on resources.  Two examples of such demonstrations: a) forming craters by dropping/throwing rocks into sand, b) using custom software to show changes in the night sky with changes in location (our Sun vs. distant suns).  If budgeting, time, and class size permit, it may also be possible for the class to take in a planetarium show at the American Museum of Natural History, or do some stargazing atop Columbia University's Astronomy department.


Angelique Davis




\end{document}