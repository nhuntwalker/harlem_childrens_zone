\section*{Motivation}

Young kids---particularly those of middle school age---are naturally awed by and enthusiastic about the sciences because science is their first real introduction into how the world works. These fresh concepts spark their imagination of what is and isn't possible in the Universe. However, as they get older they encounter a combination of dry, formulaic teaching and a lack of connection to the real world which stifles the imagination and causes interest in the sciences to crumble. This imagination is essential to experiencing the benefits of science. The damaging effects are that much more severe, where there is the societal expectation that people of color aren't cut out for science, a field dominated by people that don't look, act, or think like them. Their lack of a desire to be in that unimaginative, unengaging, and unfamiliar environment manifests in mew forms in their minds. Many conclude that science, as a whole, is boring and therefore they never aspire to develop the ability to understand science at all.\\

As a student I have experienced classroom models and teaching styles that deter students and some that engage students. For three years I have served as a teaching assistant at the college and high school levels. I have been looking for opportunities to improve my presentation in an environment of my own creation.  I realize that the Harlem Children's Zone services a large demographic of students of color. I am particularly interested in working with youngsters from the Harlem Children's Zone because I have personally witnessed a severely low presence of minorities in the hard sciences.  As such, I have designed this course to counteract disillusionment in those critical middle-school years. My curriculum does not skimp on content or rigor that is necessary to truly understand the cosmos. For six weeks I will lead my students on a fun-filled, intriguing journey through the various objects that occupy our Universe. We will tour the concepts of the night sky, and cover some of the landmark technologies that brought us knowledge of the Universe and allow the students to make their own discoveries of what is in the Universe that surrounds us. Ultimately, I want students to feel that same interest in the science of the Universe as they felt when they first saw the stars.\\




