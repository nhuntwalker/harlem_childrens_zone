\section*{Curriculum Description}
Astronomy is the study of the night sky, the Universe, and ultimately ourselves. Through astronomy we seek to answer questions of our origins, the motions of the universe, and the objects that twinkle in the dark. This course is a detailed walk through the science and technologies of the field, revealing to students the objects and concepts that have occupied the greatest human minds for centuries.\\

We begin with a brief description of the science of astronomy. I'll describe how astronomy grew from  simply tracking the stars in the sky to the science of their study. Then we'll move to the most prominent object in the night sky: the Moon. We'll discuss its dynamic history and growth into what we see today. We'll discuss its violent formation, dynamic life, and then learn to find its age.  We'll see the truth behind its phases, and connect that to what we see in the night sky throughout each month. We'll then build our own telescopes and survey its surface with our own eyes.\\

Our next stop is planets in and out of our Solar System. We'll learn what makes a planet, then how a planet is made. Some of the most advanced technology today is dedicated to finding planets, so we'll talk about those technologies as well as practice their planet-finding method. As humans, our main concern is how planets either allow for or discourage life. As such, we focus in on all the necessary components that allow life as we know it, including how planets maintain (and lose) atmospheres.\\

From there we have a natural transition to the fire in our sky, the Sun. We'll open the door to the furnace of its interior learning about what nuclear fusion is and how it works. It's constantly in a battle with itself between expansion and wholesale collapse, and we'll see how that happens. Then we'll learn how its life will end, and what that means for our planet. The Sun gives our planet all of its life, light, and energy. In fact, light is the most important thing in astronomy. It's how we know most of what we do about the Universe. As such, we'll see what light is, how it works, and learn to harness and manipulate it ourselves.\\

Our Sun is only one of the hundreds of billions of stars within our galaxy. The events of their lives characterize our night sky, and we'll discuss how each type of star lives. We will also come to see what makes stars are red, yellow, or blue (and why none are green!). We'll talk about space travel to those stars and what makes it so difficult. Then we'll build simple compressed-water rockets to mimic those that got all of our satellites into space, and will one day get astronauts to other stars. Having learned all about the stars in our sky, we'll use them to create our own constellations and stories to match.\\
%http://workshop.chromeexperiments.com/stars/

We see each star individually, but the ensemble of all of the stars in our sky are part of what constitutes our Galaxy. We will learn about the other components of our galaxy (gas, dust, etc.), and discuss its overall structure. Galaxies don't live static lives, so we'll see how nearby galaxies influence one another and give us the myriad of shapes and structures that we see in space.\\
%http://mcdonaldobservatory.org/teachers/classroom/MakeYourOwnGalaxy.html

The last stop on our journey will zoom out to our Universe as a whole. We'll discuss what we believe to be its origins, and talk about the dark mysteries that continue to elude astronomers today. We will even create our own mini universes, and learn about how their expansion illustrates what we see in our own universe. Then we head to the American Museum of Natural History for more activities, details on class topics, and a planetarium show about our dark universe.\\

Learning happens best when there is work being done inside and outside of the classroom. As such, in addition to the activities, labs, and demonstrations, I will send students home with a short question set about each week's topic, as well as encourage them to ask me questions about space, specifics of different space missions, or careers in astronomy and astrophysics. Lastly, because the beauty of the cosmos is what draws all eyes toward the night sky, I will have students search through the archives of Astronomy Picture of the Day (\url{http://apod.nasa.gov/apod/astropix.html}) each week and come up with an image that interests them, with an explanation of their interest. Through six weeks of these lessons and assignments, students will be sure to walk away with a deeper understanding of the science and contents of our night sky.

