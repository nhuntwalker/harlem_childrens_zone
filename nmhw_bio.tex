\documentclass[12pt]{article}
\usepackage{amsmath}
\usepackage{amssymb}
\usepackage[hmargin=1in,vmargin=0.1in]{geometry}

\begin{document}
\title{Nicholas Hunt-Walker Brief Biography}
\date{}
\maketitle

\vspace{-0.5in}
%Nicholas graduated from Elmont Memorial Junior-Senior High School in 2003 with honors. He went on to attend college at SUNY Binghamton, but left at the end of 2004. After a semester off from higher education, he returned to college at CUNY Queensborough Community College in the summer of 2005, majoring in business administration with the intent to become an accountant. However, after taking part in an introductory astronomy course in his second semester, he rediscovered his love for the physics he took in high school. As a result, he switched his focus toward majoring in the physical sciences, intending to become an astronomer. Along with his classes, he spent his last semester at Queensborough tutoring students in mathematics, astronomy, and accounting. Before transferring from Queensborough, Nicholas was inducted into the Phi Theta Kappa honor society.\\
Nicholas discovered his love for physics during his junior and senior years of high school at Elmont Memorial Junior-Senior High School. This continued at the City University of New York (CUNY) at Queensborough Community College, where he took his first course in astronomy. This course caused him to change from a major in business administration and transfer to CUNY York College in 2007 to pursue his Bachelor's degree in physics and mathematics. After two and a half years at York, he finished his undergraduate career with the Post-Baccalaureate Program at Columbia University, serving as a researcher in high-energy astrophysics.  In the late summer of 2010 he moved to Seattle, where he began his graduate studies in astronomy at the University of Washington (UW). He has thus far obtained his Masters of Science in astronomy (2012), and is currently in pursuit of his Ph.D.\\

While an undergraduate at York, Nicholas tutored freshman chemistry and sophomore-level physics and mathematics for 2 years. Alongside tutoring, he was active in research throughout his time at York. In the summer of 2007 he interned at the Smithsonian Institute's National Air and Space Museum in the department of Space History. While there, he aided in several local outreach events while researching how the discovery of Cosmic Microwave Background Radiation altered astronomy education. In the summer of 2008 he took part in the National Science Foundation's Research Experiences for Undergraduates (NSF-REU) program at the American Museum of Natural History (AMNH), researching the rate at which stars form in nearby galaxies. Due to his research experience and his academic record, Nicholas was awarded York College's Dr. Eugene Levin Scholarship for Excellence in the Sciences in 2008. The following summer he was an NSF-REU participant at the University of Wisconsin-Madison, looking for exotic X-ray emission in a nearby dwarf galaxy.\\

New York's astronomy community is extremely inclusive and encouraging. Because of that, Nicholas spent his last year at Columbia University in Columbia's Post-Baccalaureate Program. While there, he participated in the Rooftop Variables outreach program, assisting local high school teachers in bringing astronomy to their own campuses. He also took part in research using new gamma-ray observations to look for the high-energy remnants of stars in our galaxy. At the same time, he continued work with his undergraduate advisor at York to look at gamma-ray emission from other galaxies, based again at AMNH.\\

%After graduating from York College at the top of his class in June 2010, Nicholas moved to Seattle, WA to start on his Ph.D. in astronomy at UW. 
He was awarded the Graduate Opportunities \& Minority Achievement Program fellowship for his first year as a Ph.D-candidate at UW in 2010. The result of that was a publication on the volatile activity of the flare star AD Leo. He served as a teaching assistant (TA) for six academic quarters after that, administering labs and holding review sessions for the introductory astronomy courses. In the summer of 2011 he assisted with the maintenance of the 3.5-meter telescope's mirror at Apache Point Observatory in Sunspot, New Mexico. Afterward, he proposed a project using that observatory to search for more of the rare R Coronae Borealis stars in our galaxy, of which less than 100 are currently known. He has served as the technical coordinator for UW's on-site planetarium since summer 2012, maintaining the facilities and updating the software of UW's local version of Microsoft's World Wide Telescope. Since the 2012-2013 academic year he has conducted research on our galaxy's structure through infrared observations of some of the Milky Way galaxy's brightest, oldest stars. This work is serving as the basis for his Ph.D. thesis.\\

\end{document}