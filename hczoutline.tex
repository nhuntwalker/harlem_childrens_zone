\section*{Curriculum Outline}
\begin{itemize}
\item Elective
\item \textbf{Time:} \\50 minutes per class; 3 days per week; 1 or 2 groups per day (depending on interest).
\item \textbf{Capacity:} \\20 students maximum per class.
\item \textbf{Format and Activities:} \\ Interacting with Microsoft's WorldWide Telescope software (I maintain our University's version of the software, so there's no learning curve for me), various hands-on or interactive activities, dedicated class-long labs, chalkboard lectures, and PowerPoint lectures. I will incorporate daily take-home challenges where students will be required to solve problems, conduct their own investigations and create their own scientific data record-keeping system. 
\item \textbf{Assignments:} \\Classwork daily, homework twice per week, lab assignments over the weekends.
\item \textbf{Integrated Subjects:} \\
	The benefits of scientific study---particularly astronomy---inherently provides for the intermingling of various subjects:
	\begin{itemize}
	\item Mathematics (algebra, geometry, arithmetic, trigonometry)
	\item Physics (gravity, motion, energy, light, sound)
	\item Geology (rock types, rock dating, volcanoes, earthquakes)
	\item Chemistry (basic reactions, atmospheric gases, phases of matter)
	\item Biology (photosynthesis, DNA, eyes, water in the body)
	\end{itemize}
	
\item \textbf{Expected Outcomes:}
	\begin{itemize}
	\item Students will obtain a new or renewed appreciation for the wonders of science
	\item Students will learn how to apply basic and intermediate mathematics to practical situations
	\item Students will become familiar with the logical and integrative thought processes inherent in the Scientific Method
	\item Students will realize that humans are not omniscient and will accept the fact that questions should be asked. They will feel comfortable pursuing unknown knowledge
	\item Students will become comfortable expressing and testing their ideas
	\end{itemize}


\end{itemize}
\subsection*{Week 1 - Basic Physics and the Moon}
\begin{itemize}
\item Day 1: Pictures of the sky, gravity, object motion
	\begin{itemize}
	\item \textbf{Demonstration}: Magnetic accelerator cannon
	\item \textbf{Activity}: Build a catapult and predict where it shoots
	\end{itemize}
\item Day 2: The Moon, its origins, and its age
	\begin{itemize}
	\item \textbf{Activity}: Age dating rocks from radioactive decay
	\end{itemize}
\item Day 3: More on the Moon, it's dynamic history, building a simple telescope
	\begin{itemize}
	\item \textbf{Lab}: Build a simple reflecting telescope and look at the Moon's craters.
	\end{itemize}
\end{itemize}

\subsection*{Week 2 - Planet Formation, More on Gravity, Intro to Light}
\begin{itemize}
\item Day 1: What is a planet and how is it formed?
	\begin{itemize}
	\item \textbf{Activity}: Create a planet with its own plants and animals, atmosphere, day/year length.
	\end{itemize}
\item Day 2: How planets settle into orbits around stars, and predicting planet orbits
	\begin{itemize}
	\item \textbf{Activity}: Build a star system
	\end{itemize}
\item Day 3: Finding other planets
	\begin{itemize}
	\item \textbf{Lab}: Detecting planets with radial velocities and transits
	\end{itemize}
\end{itemize}

\subsection*{Week 3 - Life of the Sun and Much About Light}
\begin{itemize}
\item Day 1: Formation and life of the Sun
	\begin{itemize}
	\item \textbf{Activity}: Gas pressure vs elasticity showing hydrostatic equilibrium.
	\end{itemize}
\item Day 2: The Sun's light carries energy and tells us about it's composition
	\begin{itemize}
	\item \textbf{Demo}: Radiometer
	\item \textbf{Demo}: Light through a prism
	\item \textbf{Demo}: Different colors of fire and spectra
	\end{itemize}
\item Day 3: The Sun's light interacts with matter and creates all our sky color and sunsets.
	\begin{itemize}
	\item \textbf{Demo}: Shining light through a cloudy tank of water
	\item \textbf{Lab}: Solving a Light Maze
	\end{itemize}
\end{itemize}

\subsection*{Week 4 - Other Stars and Space Travel}
\begin{itemize}
\item Day 1: Wide variety of stars and their deaths
	\begin{itemize}
	\item \textbf{Demo}: Relative sizes of stars in the galaxy
	\end{itemize}
\item Day 2: Rockets and space travel
	\begin{itemize}
	\item \textbf{Outing/Lab}: Hike to Central Park.  Students build their own rockets to the stars
	\end{itemize}
\item Day 3: The truth and world history behind the constellations
	\begin{itemize}
	\item \textbf{Activity}: Create your own constellation and constellation story
	\end{itemize}
\end{itemize}

\subsection*{Week 5 - The Milky Way Galaxy and Other Galaxies}
\begin{itemize}
\item Day 1: The Milky Way in our sky and its contents in different wavelengths
	\begin{itemize}
	\item \textbf{Demo}: Views of the sky with WorldWide Telescope
	\end{itemize}
\item Day 2: \textbf{Field Trip}: Columbia University Department of Astronomy

\item Day 3: Other Galaxies 
	\begin{itemize}
	\item \textbf{Demo}: Wide variety of other galaxies and classifications
	\item \textbf{Lab}: Form a galaxy
	\end{itemize}
\end{itemize}

\subsection*{Week 6 - The Rest of the Cosmos}
\begin{itemize}
\item Day 1: Dark Matter \& Dark Energy
	\begin{itemize}
	\item \textbf{Demo}: Gravitational lensing
	\end{itemize}
\item Day 2: The Big Bang and evolution of the Cosmos
	\begin{itemize}
	\item \textbf{Lab}: Inflate a Universe
	\end{itemize}
\item Day 3: \textbf{Field Trip}: American Museum of Natural History Hayden Planetarium 

\end{itemize}
