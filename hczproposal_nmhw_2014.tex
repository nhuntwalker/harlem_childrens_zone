\documentclass[12pt]{article}
\usepackage{amsmath}
\usepackage{amssymb}
\usepackage[hmargin=0.75in,vmargin=0.5in]{geometry}
\usepackage{hyperref}
\hypersetup{pdftex,  % needed for pdflatex
  breaklinks=true,  % so long urls are correctly broken across lines
  colorlinks=true,
  urlcolor=blue,
  linkcolor=darkorange,
  citecolor=darkgreen,
  }
\begin{document}
\title{Harlem Children's Zone: Astronomy Course Proposal}
\author{Nicholas Hunt-Walker}
\maketitle

The overall objective in this proposal is to tour the contents of the night sky in 6-weeks. The primary goals will be bringing space to ground-level, and making the practice of space science achievable for all. The format will be a walk through the various objects that occupy our Universe. At each stop on our walk, we will study the details of those objects and get a deeper feel for them. While we discuss each, I will introduce the physical concepts that bring us the knowledge of the night sky that we have today.\\

Starting with the Moon, we discuss its dynamic history and growth into what we see today. We will cover the popular formation hypotheses and learn to reject all but one.  We will cover the phases of the Moon and connect that to what we see in the night sky throughout each month, and talk about how the position of our Moon causes the solar eclipses that we see.\\

Our next stop will be planets in and out of our Solar System. We will talk about how planets acquire and influence their moons, then move on to how planets form in general. Our main concern with planets as humans is how they either allow for or discourage life. As such, we focus in on all the necessary components that allow life as we know it, including how planets maintain atmospheres. Our capstone for this phase will be the ongoing search for planets outside of our Solar System. Through analysis of a combination of actual and simulated data, we will have students try to detect planets of their own around Sun-like stars.\\

From there we have a natural transition to our own Sun. Our discussion will begin with its formation, and move on to how it sustains itself through nuclear fusion, how it provides for life on our planet, and also how it can end that life. We will encounter sunspots in the solar cycle and the enormous solar flares that our planet constantly defends against. This topic will end with actual data from the Solar Dynamics Observatory and Solar Heliospheric Observatory, chronicling how the Sun's activity changes over time.\\

Our Sun is only one of the hundreds of billions of stars within our galaxy. The way in which they are born is almost as exciting as the many and varied ways in which they live and evolve. We encounter all of those, as well as their remnants after their evolution. We will also come to see what makes stars are red, yellow, or blue (and why none are green!).\\

We see each star individually, but the ensemble of all of the stars in our sky are part of what constitutes our Galaxy. We will learn about the other components of our galaxy (gas, dust, etc.), and discuss its overall structure. At the end, we will put all of this information together to come up with a picture of our own galaxy,  comparing that to what we see in our own night sky.\\

The last stop on our journey will zoom out to our Universe as a whole. We will explore clusters of galaxies and how they fit into what is known as the Large Scale Structure of our Universe. Finally, we will use actual data of distant galaxies to show that other galaxies are flying away from us, filling out our expanding Universe.\\

It is my firm belief that learning happens best when there is work being done inside and outside of the classroom. As such, in addition to the lab demonstrations, I will send students home with a short question set about each week's topic, as well as encourage them to ask me questions about space, specifics of different space missions, or careers in astronomy and astrophysics. Lastly, because the beauty of the cosmos is what draws all eyes toward the night sky, I will have students search through the archives of Astronomy Picture of the Day (\url{http://apod.nasa.gov/apod/astropix.html}) each week and come up with an image that interests them, with an explanation of their interest. Depending on timing, budgeting, and class size, we may even be able to go on a field trip to the planetarium at the American Museum of Natural History, or do some star gazing at Columbia University. Through six weeks of these lessons and assignments, students will be sure to walk away with a deeper understanding of the contents of our night sky.

\end{document}